% filepath: /mnt/d/OneDrive/FH Dortmund/Sem4/Master-Thesis/Implementation/Simulation/optimizer_pseudocode.tex

\begin{algorithm}[ht]
\caption{\texttt{Optimizer::optimize()} -- Optimization Core}
\SetAlgoLined

% Inputs and outputs
\KwIn{Model parameters, observed data, weather prediction, configuration sets}
\KwOut{Optimized configuration, power consumption, operation time}

% Step 1: Early exit checks
\If{optimizer is off or model parameter invalid or no new data}{
    \Return
}

% Step 2: Preprocessing and normalization
Normalize pixel and fps sets\;
Read and normalize weather prediction\;
Read model parameters\;
Compute actuator and sensor model coefficients\;

% Step 3: Setup Gurobi environment
Initialize Gurobi environment and model\;
Set model parameters (non-convex, output flags, etc.)\;

% Step 4: Define variables for each drone
\ForEach{drone $d$ in fleet}{
    Add variables for speed, height, fps, pixel, area, power, energy, etc.\;
    Add selector variables for discrete configurations (pixel, fps)\;
    Add binary variable for drone usage\;
}

% Step 5: Add constraints
\ForEach{drone $d$}{
    Add constraints for variable relationships (e.g., normalization, area, power)\;
    Add selector constraints (only one config per drone)\;
    Add operation time and charging constraints\;
}
Add constraint: sum of covered area equals field area\;
Add constraint: total drones used $\leq$ available\;

% Step 6: Define objective
Set objective to minimize total energy and drone usage\;

% Step 7: Optimize
Call Gurobi optimizer\;

% Step 8: Handle results
\If{solution is optimal}{
    Print and store results for each used drone\;
    Print totals (power, energy, area)\;
} \Else {
    Print error and model status\;
    \If{infeasible or unbounded}{
        Print diagnostic message\;
    }
}
Export model for diagnostics (optional)\;
\label{alg:optimizer_pseudocode}
\end{algorithm}
